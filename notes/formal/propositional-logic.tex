\section{Propositional Logic}

In order to make arguments rigorous, we need to develop a language in which we can \ctextbf{express sentences in such a way that brings out their logical structure}. 
The language we begin with is the language of \ctextbf{propositional logic}. 
It is based on propositions, or declarative sentences which one can, in principle, argue as being \ctextbf{true} or \ctextbf{false}.

Primarily, we are interested in precise declarative sentences, or statements about the behavior of computer systems, or programs. 
Not only do we want to specify such statements but we also want to \ctextbf{check whether a given program, or system, fulfils a specification at hand}. 
Thus, we need to develop a \ctextbf{calculus of reasoning} which allows us to draw conclusions from given assumptions, like initialized variables, which are reliable in the sense that they preserve truth: if all our assumptions are true, then our conclusion ought to be true as well. 
A much more difficult question is whether, \ctextbf{given any true property of a computer program, we can find an argument in our calculus that has this property as its conclusion}.

The logics we intend to design are \ctextbf{symbolic} in nature. 
We translate a certain sufficiently large subset of all English declarative sentences into strings of symbols. 
This gives us a compressed but still complete encoding of declarative sentences and allows us to concentrate on the mere mechanics of our argumentation. 
This is important since \ctextbf{specifications of systems or software are sequences of such declarative sentences}. 
It further opens up the possibility of \ctextbf{automatic manipulation} of such specifications, a job that computers just love to do 1 . 
Our strategy is to consider certain declarative sentences as being 
\footnote{In mathematical logic, an atomic formula (also known simply as an atom) is \ctextbf{a formula with no deeper propositional structure}.} \ctextbf{atomic}, or \ctextbf{indecomposable}.

Compound formulas are formed by combining the atomic formulas using the logical connectives. 
We can form more complex sentences according to the rules below:
\begin{description}
  \item[$\neg$]: $\neg \phi$ means $\phi$ not happened;
  \item[$\vee$]: $\phi \vee \psi$ means one of $\phi$ and $\psi$ happened;
  \item[$\wedge$]: $\phi \vee \psi$ means both of $\phi$ and $\psi$ happened;
  \item[$\to$]: in $\phi \to \psi$, $\phi$ is the assumption, and $\psi$ is the conclusion.
\end{description}

\subsection{Natural deduction}

In natural deduction, we have such a collection of \ctextbf{proof rules} (shown in Tab. \ref{tab:rndeduction}). 
They allow us to infer formulas from other formulas. 
By applying these rules in succession, we may infer a conclusion from a set of premises.

\begin{table}[H]
  \centering
  \caption{The rules for natural deduction}
  \vspace{0.5em}
  \begin{tabular}{c c c}
    \hline
    \thead{Operator} & \thead{Introduction} & \thead{Elimination} \\
    \hline
    $\wedge$ & 
    $\wedge i$: $\frac{\phi, \psi}{\phi \wedge \psi}$ & 
    $\wedge e_1$: $\frac{\phi \wedge \psi}{\phi}$, 
    $\wedge e_2$: $\frac{\phi \wedge \psi}{\psi}$ \\

    $\vee$ & 
    $\vee i_1$: $\frac{\phi}{\phi \vee \psi}$, 
    $\vee i_2$: $\frac{\psi}{\phi \vee \psi}$ & 
    $\vee e$: $\frac{\phi \vee \psi, \boxed{\phi \dots \chi}, \boxed{\psi \dots \chi}}{\chi}$ \\

    $\to$ &
    $\to i$: $\frac{\boxed{\phi \dots \psi}}{\phi \to \psi}$ &
    $\to e$: $\frac{\phi, \phi \to \psi}{\psi}$\\

    $\neg$ &
    $\neg i$: $\frac{\boxed{\phi \dots \bot}}{\neg \phi}$ &
    $\neg e$: $\frac{\phi, \neg\phi}{\bot}$ \\

    $\bot$ &
    no introduction rule for $\bot$ &
    $\bot e$: $\frac{\bot}{\phi}$\\

    $\neg\neg$ &
    no introduction rule for $\neg\neg$ &
    $\neg\neg e$: $\frac{\neg\neg\phi}{\phi}$ \\
    \hline
  \end{tabular}
  \label{tab:rndeduction}
\end{table}

Some useful derived rules:

\begin{table}[H]
  \centering
  \begin{tabular}{c c}
    $MT$: $\frac{\phi \to \psi, \neg \psi}{\neg \phi}$ &
    $\neg\neg i$: $\frac{\phi}{\neg\neg \phi}$ \\
    $PBC$: $\frac{\boxed{\neg \phi \dots \bot}}{\phi}$ &
    $LEM$: $\frac{}{\phi \vee \neg \phi}$ \\
  \end{tabular}
\end{table}

\subsection{Propositional logic as a formal language}

Here we use a formal language to describe propositional logic.
A defining grammar in \ctextbf{Backus Naur form (BNF)} of the logic is presented as follows:
\begin{align*}
  \phi, \psi ::= p | (\neg \phi) | (\phi \wedge \psi) | (\phi \vee \psi) | (\phi \to \psi)
\end{align*}
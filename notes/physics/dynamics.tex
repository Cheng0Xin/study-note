\section{Dynamics}

\subsection{Introduction to Dynamics}

\textbf{Some important concepts}

Dynamics has two distinct parts: 
\begin{itemize}
  \item \ctextbf{kinematics}, 
    which is the study of motion without reference to the 
    forces which cause motion, and
  \item \ctextbf{kinetics} which relates the action of forces 
    on bodies to their resulting motions.
\end{itemize}

Newton's Laws:
\begin{itemize}
  \item[\textbf{I}] A particle remains at rest or continues to move with uniform
    velocity (in straight line with a constant speed) if there is no unbalanced 
    force acting on it.
  \item[\textbf{II}] The acceleration of a particle is proportional to the
    resultant force acting on it and is in the direction of this force.
  \item[\textbf{III}] The forces of action and reaction between interacting
    bodies are equal in magnitude, opposite in direction, and collinear.
\end{itemize}

\subsection{Kinematics of Particles}

\textbf{Some important concepts}

If the particle is confined to a specified path, as with a bead sliding along a
fixed wire, its motion is said t be constrained. 
If there are no physical guides, the motion is said to be unconstrained.

Choice of coordinates:
\begin{itemize}
  \item rectangular coordinates (\ctextbf{Cartesian coordinates}), $(x, y, z)$;
  \item \ctextbf{cylindrical coordinates}, $(r, \theta, z)$;
  \item \ctextbf{spherical coordinates}, $(R, \theta, \phi)$;
\end{itemize}

The motion of particle can be described by using coordinates measured from fixed
reference axes (\ctextbf{absolute-motion analysis}) or by using coordinates 
measured from moving reference axes (\ctextbf{relative-motion analysis}).

\textbf{Some concepts for} \ctextbf{Rectilinear Motion}:

\textbf{Some terms}:
\begin{itemize}
  \item $t$: time;
  \item $s$: distance;
  \item $v$: velocity;
  \item $a$: acceleration;
  \item $\Delta{t}$: interval;
  \item $\Delta{s}$: displacement.
\end{itemize}

The velocity $v$ is \ctextit{the time rate of change} of the position coordinate $s$.
\begin{align}
  v = \frac{ds}{dt} = \lim_{\Delta{t} \to 0}\frac{\Delta{s}}{\Delta{t}} = \dot{s}
  \label{eq:velocity-displacement}
\end{align}

\begin{align}
 s 
\end{align}

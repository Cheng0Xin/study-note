\section{Analysis}

\subsection{Logic}

This part is concluded from 
\begin{itemize}
  \item \emph{Mathematical Analysis I} written by Vladimir A. Zorich, and
  \item \emph{Discrete Mathematics and its Applications} written by Kenneth H. Rosen.
\end{itemize}

\crule

\subsubsection{Propositional Logic}

A \ctextbf{proposition} or \ctextbf{statement} is a \ctextbf{declarative} 
sentence that is either true or false, but not both.
It is the basic building blocks of logic.
The \ctextbf{propositional variables} are denoted by letters,
while the \ctextbf{truth values} of them, 
i.e. true and false, are denoted by T, F respectively.

The area of logic that deals with propositions is called 
\ctextbf{propositional logic} or \ctextbf{propositional calculus}.

New propositions, called \ctextbf{compound propositions}, are 
\ctextit{formed from existing propositions}using 
\ctextbf{logical operators (connectives)}.

The BNF of propositions is, 
\begin{align*}
  proposition := p_1 | p_2 | 
    \textbf{a}: \neg{p_1} | 
    \textbf{b}: p_1 \wedge p_2 | 
    \textbf{c}: p_1 \vee p_2 | 
    \textbf{d}: p_1 \rightarrow p_2 | 
    \textbf{e}: p_1 \leftrightarrow p_2
\end{align*}
where, \emph{a} is negation, \emph{b} is conjunction, \emph{c} is disjunction,
\emph{d} is conditional statement or implication.
\emph{e} is bi-conditional statement or bi-implication.
In \emph{d}, $p_1$ is hypothesis, $p_2$ is conclusion.
$p_2 \rightarrow p_1$ is the \ctextbf{converse} of \emph{d},
and $\neg{p_2} \rightarrow \neg{p_1}$ is the \ctextbf{contrapositive} of \emph{d}.
The statement and its contrapositive are \ctextbf{equivalent}.

According to the logical operators, we can get bit operations, 
i.e. \textbf{OR} $\vee$, \textbf{AND} $\wedge$ and \textbf{XOR} $\oplus$.

\crule

\subsubsection{Application of Propositional Logic}

\begin{itemize}
  \item[A.] Translating English Sentence. 
    Translating sentences into compound statements removes the ambiguity.
  \item[B.] System Specification. 
    Translating sentences in natural language into logical expression is
    an essential part of specifying both hardware and software system.
  \item[C.] Other applications, such as Boolean Searches, Logic Puzzles, 
    and Logic Circuits.
\end{itemize}

\crule

\subsubsection{Propositional Equivalences}

There three types of compound propositions,
\begin{itemize}
  \item \ctextbf{Tautology}: a compound proposition that is 
    \ctextit{always true}, 
    no matter what the truth values of the propositional variables 
    that occur in it.
  \item \ctextbf{Contradiction}: a compound proposition that is 
    \ctextit{always false}.
  \item \ctextbf{Contingency}: a compound proposition that is neither a 
    tautology nor a contradiction.
\end{itemize}

Compound propositions that have the same truth values in all possible cases are 
called \ctextbf{logically equivalent}. 
A formal definition of \emph{Logical Equivalence} is that
the compound propositions $p$, $q$ are called logically equivalent if 
\ctextit{$p \leftrightarrow q$ is a tautology}. 
The notation $p \equiv q$ denotes that $p$ and $q$ are logically equivalent.

\begin{table}[H]
  \centering
  \caption{Logical Equivalences (Part I)}
  \vspace{0.5em}
  \begin{tabular}{c c}
    \toprule
    \midrule
    \thead{Equivalence} & \thead{Name} \\
    \midrule
    $\neg(p \wedge q) \equiv \neg{p} \vee \neg{q}$, 
    $\neg(p \vee q) \equiv \neg{p} \wedge \neg{q}$ & De Morgan's laws \\
    $p \wedge T \equiv p$, $p \vee F \equiv p$ & Identity laws \\
    $p \vee T \equiv T$, $p \wedge F \equiv F$ & Domination laws \\
    $p \vee p \equiv p$, $p \wedge p \equiv p$ & Idempotent laws \\
    $\neg{(\neg{p})} \equiv p$ & Double negation law \\
    $p \vee q \equiv q \vee p$,
    $p \wedge q \equiv q \wedge p$ & Commutative laws \\
    $p \vee q \vee r \equiv p \vee (q \vee r)$, 
    $p \wedge q \wedge r \equiv p \wedge (q \wedge r)$ & Associative laws \\
    $p \vee (q \wedge r) \equiv (p \wedge q) \vee (p \wedge r)$,
    $p \wedge (q \vee r) \equiv (p \vee q) \wedge (p \vee r)$ 
    & Distributive laws \\
    $p \vee (p \wedge q) \equiv p$,
    $p \wedge (p \vee q) \equiv p$ & Absorption laws \\
    $p \vee \neg{p} \equiv T$, $p \wedge \neg{q} \equiv F$ & Negation laws \\
    \toprule
  \end{tabular}
\end{table}

\begin{table}[H]
  \centering
  \caption{Logical Equivalences (Part II)}
  \vspace{0.5em}
  \begin{tabular}{c c}
    \toprule
    \midrule
    \thead{No.} & \thead{Equivalence} \\
    \midrule
    1 & $p \rightarrow q \equiv \neg{p} \vee q$ \\
    2 & $p \rightarrow q \equiv \neg{q} \rightarrow \neg{p}$ \\
    3 & $(p \rightarrow q) \wedge (p \rightarrow r) \equiv p \rightarrow (q \wedge r)$ \\
    4 & $(p \rightarrow q) \vee (p \rightarrow r) \equiv p \rightarrow (q \vee r)$ \\
    5 & $(p \rightarrow r) \wedge (q \rightarrow r) \equiv (p \vee q) \rightarrow r$ \\
    6 & $(p \rightarrow r) \vee (q \rightarrow r) \equiv (p \wedge q) \rightarrow r$ \\
    7 & $p \leftrightarrow q \equiv (p \rightarrow q) \wedge (q \rightarrow p)$ \\
    8 & $p \leftrightarrow q \equiv (p \wedge q) \vee (\neg{p} \wedge \neg{q})$ \\
    \toprule
  \end{tabular}
\end{table}

We will sometimes use the notation $\vee^{n}_{j=1}p_j$ for 
$p_1 \vee p_2 \vee \dots \vee p_n$
and $\wedge^{n}_{j=1}p_j$ for
$p_1 \wedge p_2 \wedge \dots \wedge p_n$.

A proposition in a compound proposition can \ctextit{be replaced by}
a compound proposition that is \ctextit{logically equivalent to it}
without changing the truth value of the original compound proposition.

A compound proposition is \ctextbf{satisfiable}  
\ctextit{if these is an assignment} of truth values to its variables
that makes it true.
A compound proposition is \ctextbf{unsatisfiable} \ctextit{if and only if}
its negation is a tautology.
When we find \ctextit{a particular assignment} of truth values that makes a 
compound proposition true, we have shown that it is satisfiable; 
such an assignment is called a \ctextbf{solution} of this 
\ctextit{particular satisfiability problem}.

\crule

\subsubsection{Remarks on Proofs} 

\begin{itemize}
  \item[I.] The proof of such a proposition consists of 
    \ctextit{constructing a chain} 
    $A \Rightarrow C_1 \Rightarrow \dots \Rightarrow C_n \Rightarrow B$
    of implications, each element of which is either an axiom or 
    a previously proved proposition.
  \item[II.] In \ctextbf{proof by contradiction} we shall also use 
    \ctextbf{law of excluded middle}, by virtue of which the statement
    $A \wedge \neg{A}$ is considered \ctextit{true independently of the specific
    content} of the statement A.
\end{itemize}

\crule

\subsubsection{Notation System}

\subsection{Set}

\subsubsection{Notation System}

\subsection{Function}

\subsubsection{Notation System}

\subsection{Fourier Series}

Fourier series are infinite series that \ctextit{represent periodic functions} in terms of cosines and sines.
Given a \ctextbf{trigonometric series} shown in Eq. \ref{eq:tri-series}, the $a_i$ and $b_i$ are the constants, called \ctextbf{coefficients} of the series.

\begin{align}
  a_0 + a_1 \cos x + b_1 \sin x + a_2 \cos 2x + b_2 \sin 2x + \dots = a_0 + \sum^\infty_{n=1}(a_n \cos nx + b_n \sin nx)
  \label{eq:tri-series}
\end{align}

Suppose that a given function $f(x)$ of period $2\pi$ can be represented by Eq. \ref{eq:tri-series}, then we have Eq. \ref{eq:fx-tri-series} which is formally called \ctextbf{Fourier Series} of function $f(x)$.

\begin{align}
  f(x) = a_0 + \sum^\infty_{n=1}(a_n \cos nx + b_n \sin nx)
  \label{eq:fx-tri-series}
\end{align}

The calculation of coefficients are followed the formulas shown in Eq. \ref{eq:fourier-calcu}.

\begin{align}
  \begin{cases}
    a_0 &= \frac{1}{2\pi}\int_{-\pi}^{\pi} f(x) dx \\
    a_n &= \frac{1}{\pi}\int_{-\pi}^{\pi} f(x)\cos nx dx, (n \geq 1) \\
    b_n &= \frac{1}{\pi}\int_{-\pi}^{\pi} f(x)\sin nx dx, (n \geq 1)
  \end{cases}
  \label{eq:fourier-calcu}
\end{align}
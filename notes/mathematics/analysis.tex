\section{Analysis}

\subsection{Logic}

This part is concluded from 
\begin{itemize}
  \item \emph{Mathematical Analysis I} written by Vladimir A. Zorich, and
  \item \emph{Discrete Mathematics and its Applications} written by Kenneth H. Rosen.
\end{itemize}

\crule

\subsubsection{Propositional Logic}

A \ctextbf{proposition} or \ctextbf{statement} is a \ctextbf{declarative} 
sentence that is either true or false, but not both.
It is the basic building blocks of logic.
The \ctextbf{propositional variables} are denoted by letters,
while the \ctextbf{truth values} of them, 
i.e. true and false, are denoted by T, F respectively.

The area of logic that deals with propositions is called 
\ctextbf{propositional logic} or \ctextbf{propositional calculus}.

New propositions, called \ctextbf{compound propositions}, are 
\ctextit{formed from existing propositions}using 
\ctextbf{logical operators (connectives)}.

The BNF of propositions is, 
\begin{align*}
  proposition := p_1 | p_2 | 
    \ctextbf{a}: \neg{p_1} | 
    \ctextbf{b}: p_1 \wedge p_2 | 
    \ctextbf{c}: p_1 \vee p_2 | 
    \ctextbf{d}: p_1 \rightarrow p_2 | 
    \ctextbf{e}: p_1 \leftrightarrow p_2
\end{align*}
where, 
\begin{itemize}
  \item[a.] negation.
  \item[b.] conjunction.
  \item[c.] disjunction.
  \item[d.] conditional statement or implication.
    $p_1$ is hypothesis, $p_2$ is conclusion.
    $p_2 \rightarrow p_1$ is the \ctextbf{converse} of d,
    and $\neg{p_2} \rightarrow \neg{p_1}$ is the \ctextbf{contrapositive} of d.
    The statement and its contrapositive are \ctextbf{equivalent}.
  \item[e.] bi-conditional statement or bi-implication.
\end{itemize}

According to the logical operators, we can get bit operations, 
i.e. \textbf{OR} $\vee$, \textbf{AND} $\wedge$ and \textbf{XOR} $\oplus$.

\crule

\subsubsection{Remarks on Proofs} 

\begin{itemize}
  \item[I.] The proof of such a proposition consists of 
    \ctextit{constructing a chain} 
    $A \Rightarrow C_1 \Rightarrow \dots \Rightarrow C_n \Rightarrow B$
    of implications, each element of which is either an axiom or 
    a previously proved proposition.
  \item[II.] In \ctextbf{proof by contradiction} we shall also use 
    \ctextbf{law of excluded middle}, by virtue of which the statement
    $A \wedge \neg{A}$ is considered \ctextit{true independently of the specific
    content} of the statement A.
\end{itemize}

\crule

\subsubsection{Notation System}

\subsection{Set}

\subsubsection{Notation System}

\subsection{Function}

\subsubsection{Notation System}

\subsection{Fourier Series}

Fourier series are infinite series that \ctextit{represent periodic functions} in terms of cosines and sines.
Given a \ctextbf{trigonometric series} shown in Eq. \ref{eq:tri-series}, the $a_i$ and $b_i$ are the constants, called \ctextbf{coefficients} of the series.

\begin{align}
  a_0 + a_1 \cos x + b_1 \sin x + a_2 \cos 2x + b_2 \sin 2x + \dots = a_0 + \sum^\infty_{n=1}(a_n \cos nx + b_n \sin nx)
  \label{eq:tri-series}
\end{align}

Suppose that a given function $f(x)$ of period $2\pi$ can be represented by Eq. \ref{eq:tri-series}, then we have Eq. \ref{eq:fx-tri-series} which is formally called \ctextbf{Fourier Series} of function $f(x)$.

\begin{align}
  f(x) = a_0 + \sum^\infty_{n=1}(a_n \cos nx + b_n \sin nx)
  \label{eq:fx-tri-series}
\end{align}

The calculation of coefficients are followed the formulas shown in Eq. \ref{eq:fourier-calcu}.

\begin{align}
  \begin{cases}
    a_0 &= \frac{1}{2\pi}\int_{-\pi}^{\pi} f(x) dx \\
    a_n &= \frac{1}{\pi}\int_{-\pi}^{\pi} f(x)\cos nx dx, (n \geq 1) \\
    b_n &= \frac{1}{\pi}\int_{-\pi}^{\pi} f(x)\sin nx dx, (n \geq 1)
  \end{cases}
  \label{eq:fourier-calcu}
\end{align}
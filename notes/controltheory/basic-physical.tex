\section{Mathematical Methods of Classical Mechanics}

\subsection{Experimental Facts}

\subsubsection{The principles of relativity and determinacy}

\begin{enumerate}[label=(\roman*)]
  \item Space (three dimensional and euclidean) and time (one dimensional);
  \item Galileo's principle of relativity: the laws are the same in all inertial coordinate systems, remain inertial after uniform rectilinear motion;
  \item Newton' principle of determinacy: the initial state (position and velocity) determines all its motion.
\end{enumerate}

\subsubsection{The galilean group and Newton's equations}

\paragraph{Notion} 
\begin{itemize}
  \renewcommand{\labelitemi}{$\dagger$}
  \item $\mathds{R}$ denotes real number, $\mathds{R}^n$ denotes n-dimensional real vector-space,
  \item $A^n$ denotes affine n-dimensional space (compared with $\mathds{R}^n$, $A^n$ has no fixed origin),
  \item group $\mathds{R}^n$ acts on $A^n$ as the group of parallel displacements: $a \rightarrow a + b, a \in A^n, b \in \mathds{R}^n, a + b \in A^n$,
  \item a euclidean structure on $\mathds{R}^n$ is a positive definition symmetric bilinear form called a scalar product, $\rho(x, y)=||x - y||$.
\end{itemize}

\paragraph{Galilean structure}
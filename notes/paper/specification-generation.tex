\section{Specification Generation}

\crule

\subsection{1994- Automatic Translation of Natural Language System 
  Specifications into Temporal Logic}

\textbf{Reference:} \cite{nelken1996automatic}.

\textbf{Traditional way to formulate the specification:}
The formulation of specifications often becomes a two stage processes:
\begin{itemize}
  \item Specifications are written in natural language (NL).
  \item These specifications are manually translated into temporal logic (TL).
    This is done according to the intuitive understanding of the translator,
    who has to contend with the impression and ambiguity of NL and with subtle 
    interpretation of TL formula.
\end{itemize}

\textbf{Discourse Representation Theory (DRT).}

\crule

\subsection{2003- On the interplay between consistency, 
  completeness, and correctness in requirements evolution}

\textbf{Reference:} \cite{zowghi2003interplay}

\textbf{The three Cs:}
The specification should maintain \ctextbf{consistency}, \ctextbf{completeness} 
and \ctextbf{correctness}, where
\begin{itemize}
  \item consistency refers to situations where a specification 
    \ctextit{contains no internal contradictions}, whereas
  \item completeness refers to situations where a specification 
    \ctextit{entails everything} that is desired 
    to hold in a certain context.
  \item correctness:
    (a) From a formal point of view, correctness is usually meant to be 
      the \ctextit{combination of consistency and completeness}.
    (b) From a practical point of view, correctness can be more 
      pragmatically defined as \ctextit{satisfiability of 
      certain business goals}.
\end{itemize}


\section{Behavior Tree}

\crule

\subsection{2017- Evolutionary Behavior Tree Approaches for 
  Navigating Platform Games}

\textbf{Reference:} \cite{nicolau2016evolutionary}

\textbf{Purpose:}
\begin{itemize}
  \item Basic on \ctextit{the Mario AI benchmark platform}.
  \item Deal with \ctextbf{navigation} and \ctextbf{reactiveness} 
    by using \ctextit{$A^*$ approach}.
\end{itemize}

The paper showed the applicability of \ctextbf{behavior trees (BTs)} as 
representations for \ctextbf{evolutionary computation}, 
and their flexibility for incorporations of diverse algorithms to 
deal with specific aspects of bot control in game environment.

They investigated the use of \ctextbf{grammatical evolution (GE)} to evolve BTs.

Nodes in BT can return a success or failure (\emph{True} or \emph{False}) value 
to their parent node and are divided into two major categories:
\begin{itemize}
  \item Control nodes:
    \begin{itemize}
      \item \textbf{Sequence}, execute children nodes from left to right until 
        one return failure,
      \item \textbf{Selector}, execute children nodes until one returns success,
      \item \textbf{Filter}, include \{ loops, running children nodes until 
        failure \}.
    \end{itemize}
  \item Leaf nodes;
\end{itemize}

\subsection{2006- Introducing Grammar Based Extensions for 
  Grammatical Evolution}

\textbf{Reference:} \cite{nicolau2006introducing}
\section{Digital Twin}

\crule

\subsection{2020, Combining heterogeneous digital human simulations: presenting 
  a novel co-simulation approach for incorporating different character 
  animation technologies}

\textbf{Reference:} \cite{gaisbauer2020combining}

A variety of simulation techniques and tools are available, ranging from 
\vspace{-.5cm}
\begin{itemize}
  \item \ctextbf{motion-capture-based animation systems}, and
  \item \ctextbf{deep learning to physics-based motion synthesis}.
\end{itemize}

\clanewdef{Functional Mock-up Interface (FMI)}{
  The FMI defines an interface to be implemented by an executable called an 
  \ctextbf{Functional Mock-up Unit (FMU)}. The FMI functions are used (called)
  by a simulation environment to create one or more instances of the FMU and to 
  simulate them, typically together with other models.
}

\textbf{Purpose:}
Inspired by the FMI standard, a novel framework to incorporate multiple digital 
human simulation approaches from multiple domains is presented. 
In particular, the paper introduces the overall concept of the so-called 
\ctextbf{Motion Model Units}, as well as its underlying technical architecture.
As main contribution, a novel \ctextbf{co-simulation} for the 
orchestration of multiple digital human simulation approaches is presented.

\textbf{Traditional motion synthesis technologies:}
\vspace{-.5cm}
\begin{itemize}
  \item \ctextbf{Data-driven} approaches: Barycentric-, K-Nearest-Neighbor-,
    Radial Basis Function- interpolation and Inverse blending.
    \textbf{Disadvantages:} the approaches only cover the range given 
    by the underlying data, however, generating naturally looking results.
  \item \ctextbf{Model-based} approaches: trajectory optimization and 
    reinforcement learning.
    \textbf{Disadvantages:} the approaches are generic but also more difficult
    to parameterize. 
\end{itemize}

\crule
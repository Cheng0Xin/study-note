\section{Hybrid System: Syntax and Semantics of Continuous Program}

The differential equations with their evolution domains will ultimately need a way of interfacing with discrete computer control programs, 
because hybrid system models of cyber-physical systems combine discrete dynamics and continuous dynamics.
Before we talking about the syntax and semantics, the naming conventions is presented in Tab. \ref{tab:nconvention}.
\begin{table}[H]
  \centering
  \caption{Naming conventions of the syntax and semantics}
  \vspace{0.5em}
  \begin{tabular}{l l l}
    \hline
    \thead{No.} & \thead{Letters} & \thead{Convention} \\
    \hline
    1 & $x,y,z$ & variables \\
    2 & $e, \hat{e}$ & terms \\
    3 & $P, Q$ & formulas \\
    4 & $\alpha, \beta$ & programs \\
    5 & $c$ & constant symbols \\
    6 & $f, g, h$ & function symbols \\
    7 & $p, q, r$ & predicate symbols \\
    \hline
  \end{tabular}
  \label{tab:nconvention}
\end{table}

The first element of the syntax of hybrid programs is purely continuous programs.
\clanewdef{Continuous Program}{def:socprogram}{
  Layer 1 of \textbf{hybrid programs (HPs)} comprises 
  \textbf{continuous programs}.
  Continuous programs are of the form
  \begin{align*}
    \alpha ::= x' = e \& Q \text{,}
  \end{align*}
} 
where $x$ is a variable, 
$e$ any term possibly containing $x$,
and $Q$ a formula of first-order logic of real arithmetic.


\clanewdef{Terms}{def:terms}{
  A term $e$ is a polynomial term defined by the grammar 
  (where $e, \hat{e}$ are terms, $x$ is a variable, and $c$ is a rational number constant)
  \begin{align*}
    e, \hat{e} ::= x | c | e + \hat{e} | e \cdot \hat{e}
  \end{align*}
}

\clanewdef{Formulas of first-order logic of real arithmetic}{def:foflorarithmetic}{
  The formulas of first-order logic of real arithmetic are defined by the following grammar 
  (where $P, Q$ are formulas of first-order logic of real arithmetic, $e, \hat{e}$ are terms, and $x$ is a variable):
  \begin{align*}
    P, Q ::= e \geq \hat{e} | e = \hat{e} | \neg P | P \cap Q | P \cup Q | P \rightarrow Q | P \leftrightarrow Q | \forall x P | \exists x P
  \end{align*}
}

\clanewdef{Semantics of continuous programs}{def:seocprogram}{
  The state $\nu$ is reachable from initial state $\omega$ by the continuous program 
  $x_1'=e_1,\dots, x_n'=e_n \& Q$ iff there is a solution $\phi$ of some duration 
  $r \geq 0$ along $x_1'=e_1,\dots, x_n'=e_n \& Q$ from state $\omega$ to state $\nu$, i.e., a function $\phi : [0, r] \to \mathscr{S}$ such that:
  \begin{itemize}
    \item initial and final states match: $\phi(0) = \omega, \phi(r) = v$;
    \item $\phi$ respects the differential equations: 

    For each variable $x_i$, the value $\phi(\zeta)[x_i] = \phi(\zeta)(x_i)$ of $x_i$ at state $\phi(\zeta)$ is continuous in $\zeta$ on $[0, r]$ and, 
    if $r>0$, has a time-derivative of value $\phi(\zeta)[e_i]$ at each time $\zeta \in [0, r]$, i.e., 
    \begin{align*}
      \frac{d\phi(t)(x_i)}{dt}(\zeta) = \phi(\zeta)[e_i]
    \end{align*}
    \item the value of other variables $y \notin \{x_1, \dots, x_n\}$ remains constant throughout the continuous evolution, that is $\phi(\zeta)[y] = \omega[y]$ for all times $\zeta \in [0, r]$;
    \item and $\phi$ respects the evolution domain at all times: $\phi(\zeta) \in [Q]$ for each $\zeta \in [0, r]$.
  \end{itemize}
}

\clanewdef{Semantics of terms}{def:soterms}{
  The value of term $e$ in state $\omega \in \zeta$ is a real number denoted $\omega[e]$ and is defined by induction on the structure of $e$:
  \begin{align*}
    &\omega[x] = \omega(x) \text{ if } x \text{ is a variable} \\
    &\omega[c] = c \text{ if } c \in Q \text{ is a rational constant} \\
    &\omega[e+\hat{e}] = \omega[e] + \omega[\hat{e}] \\
    &\omega[e \cdot \hat{e}] = \omega[e] \cdot \omega[\hat{e}]
  \end{align*}
}

\clanewdef{First-order logic semantics}{def:flsemantics}{
  The first-order formula $P$ is ture in state $\omega$, 
  is writtern $\omega \vDash P$, and is defined inductively as follows:
  \begin{itemize}
    \item $\omega \vDash e = \hat{e}$ iff $\omega[e] = \omega[\hat{e}]$
    That is, an equation $e=\hat{e}$ is true in a state $\omega$ 
    iff the term $e$ and $\hat{e}$ evaluate to the same number in $\omega$ according to Def. \ref{def:soterms}.
    \item $\omega \vDash e \geq \hat{e}$ iff $\omega[e] \geq \omega[\hat{e}]$.
    \item $\omega \vDash \neg P$ iff $\omega \nvDash P$. 
    \item $\omega \vDash P \cap Q$ iff $\omega \vDash P$ and $\omega \vDash Q$.
    \item $\omega \vDash P \cup Q$ iff $\omega \vDash P$ or $\omega \vDash Q$.
    \item $\omega \vDash P \rightarrow Q$ iff $\omega \nvDash P$ or $\omega \vDash Q$.
    \item $\omega \vDash P \leftrightarrow Q$ iff 
    ($\omega \vDash P$ and $\omega \vDash Q$) or 
    ($\omega \nvDash P$ and $\omega \nvDash Q$).
    \item $\omega \vDash \forall x P$ iff $\omega^d_x \vDash P$ for all $d \in \mathds{R}$. (Evolution of $\omega^d_x$ is defined in Eq. \ref{eq:omegadx})
    \item $\omega \vDash \exists x P$ iff $\omega^d_x \vDash P$ for some $d \in \mathds{R}$.
  \end{itemize}
}

\begin{align}
  \omega^d_x(y) = 
  \begin{cases} 
    d \text{ if } y \text{ is the modified variable } x \\
    \omega(y) \text{ if } y \text{ is another variable}
  \end{cases}
  \label{eq:omegadx}
\end{align}

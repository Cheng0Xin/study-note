\subsection{Differential Equations and Domains}

\subsubsection{The Meaning of Differential Equations}

\textbf{Ordinary Differential Equation (ODE):} 
Let $f: D \to \mathds{R}^n$ be a function on a 
$\text{domain } D \subseteq \mathds{R} \times \mathds{R}^n$,
i.e., an open and connected subset.
The function $Y: J \to \mathds{R}^n$ is a \ctextbf{solution} on 
interval $J \subseteq \mathds{R}$ of the \ctextbf{initial value problem}
  \begin{align*}
    &y'(t) = f(t, y) \\
    &y(t_0) = y_0
  \end{align*}
  with ODE $y' = f(t, y)$, 
if, for all times $t \in J$: 
\vspace{-.5cm}
\begin{enumerate}
  \item solution $Y$ is in the domain $(t, Y(t)) \in D$; 
  \item time-derivative $Y'(t)$ exists and is $Y'(t) = f(t, Y(t))$; and
  \item initial value $Y(t_0)=y_0$ is respected at the initial time, also
    $t_0 \in J$.
\end{enumerate}

To solve the equations, the infinite continuous line will be 
\ctextbf{discretized} into the finite discrete points.
If the equation is quite well behaved,
the discretizations would approach the true continuous solutions as the 
$\Delta$ gets smaller.
However, no matter how small a discretization $\Delta > 0$ we choose,
that discretization will be arbitrarily \ctextit{far away} 
from the true continuous solution for large $t$.

\subsubsection{Domains of Differential Equations}

When and how do physics and cyber elements interact?
\vspace{-.5cm}
\begin{itemize}
  \item Physics might evolve and the cyber elements 
    \ctextit{interrupt physics periodically} to make measurements of 
    the current state of the system in order to decide what to do next 
    (\ctextbf{Periodical Runtime Model}).
  \item Or the physics might \ctextit{trigger certain conditions or events} 
    that cause cyber elements to compute their respective responses to 
    these events (\ctextbf{Event-Driven Runtime Model}).
\end{itemize}

The model needs to specify when physics \underline{stops evolving} to 
give cyber a chance to perform its task.
This information is what is called an \ctextbf{evolution domain $Q$} of 
a differential equation,
which describes a \underline{region} that the system \ctextit{cannot leave} 
while following that particular continuous mode of the system.

\textbf{Evolution Domain Constraint:} 
A differential equation $x' = f(x)$ with \textbf{evolution domain Q} 
is denoted by $x' = f(x) \& Q$ using a conjunctive notation (\&) 
between the differential equation and its evolution domain.
This notation $x' = f(x) \& Q$ signifies that the system obeys 
both the differential equation $x' = f(x)$ and the evolution domain $Q$. 
The system follows this differential equation for 
\ctextit{any duration while inside the region $Q$}, 
but is \ctextit{never allowed to leave} the region described by $Q$.
So the system evolution has to stop while the state is not in $Q$.

\subsubsection{Syntax of Continuous Programs}

\textbf{Continuous Program:} 
Layer 1 of hybrid programs (HPs) comprises \ctextbf{continuous programs}. 
If $x$ is a variable, $e$ is any term possibly containing $x$,
and $Q$ a formula of \underline{first-order logic} of real arithmetic, 
then continuous programs are of the form,
\begin{align*}
  \alpha ::= x' = e \& Q
\end{align*}

\textbf{Term:} A term $e_i$ is a \ctextbf{polynomial term} defined by 
the grammar (where $e_0$, $e_1$ are terms, $x$ is a variable, 
and $c$ is a rational number constant),
\begin{align*}
  e_0, e_1 = x | c | e_0 + e_1 | e_0 \cdot e_1
\end{align*}

\textbf{Formulas of First-order Logic of Leal Arithmetic:}
The formulas of first-order logic of real arithmetic are defined by the 
following grammar (where $P$,$Q$ are formulas of first-order logic of 
real arithmetic, $e_0$, $e_1$ are terms, and $x$ is a variable),
\begin{align*}
  P, Q = e_0 \geq e_1 | e_0 = e_1 | \neg{P} | P \vee Q | P \wedge Q | 
    P \rightarrow Q | P \leftrightarrow Q | \forall{x}P | \exists{x}P
\end{align*}

\subsubsection{Semantics of Continuous Programs}

\textbf{Semantics of Terms:}
The value of term $e$ in state $\omega \in S$ is a real number denoted 
$\omega\llbracket e \rrbracket$ and is defined by \ctextbf{induction} on the 
structure of $e$:
\begin{align*}
  \omega\llbracket x \rrbracket &= \omega(x) 
    \text{, if $x$ is a variable} \\
  \omega\llbracket c \rrbracket &= c 
    \text{, if $c \in \mathds{Q}$ is a rational constant}\\
  \omega\llbracket e_0 + e_1 \rrbracket &= \omega\llbracket e_0 \rrbracket + 
    \omega\llbracket e_1 \rrbracket \\
  \omega\llbracket e_0 \cdot e_1 \rrbracket &= \omega\llbracket e_0 \rrbracket 
    \cdot \omega\llbracket e_1 \rrbracket \\
\end{align*}

A state $(\omega: \mathscr{V} \to \mathds{R}) \in \mathscr{S}$ 
is a \ctextbf{mapping} from variables to real numbers, where
$\mathscr{V}$ is the set of all variables, 
and $\mathscr{S}$ is the set of states.

\textbf{First-order Logic Semantics:}
The first-order formula $P$ is true in state $\omega$, 
is written $\omega \vDash P$, and is defined inductively as follows:
\vspace{-.5cm}
\begin{itemize}
  \item $\omega \vDash e = \hat{e}$ iff 
    $\omega \llbracket e \rrbracket = \omega\llbracket\hat{e}\rrbracket$
  \item $\omega \vDash e \geq \hat{e}$ iff
    $\omega\llbracket e \rrbracket \geq \omega\llbracket \hat{e} \rrbracket$.
  \item $\omega \vDash \neg P$ iff 
    $\omega \nvDash P$. 
  \item $\omega \vDash P \cap Q$ iff 
    $\omega \vDash P$ and $\omega \vDash Q$.
  \item $\omega \vDash P \cup Q$ iff 
    $\omega \vDash P$ or $\omega \vDash Q$.
  \item $\omega \vDash P \rightarrow Q$ iff 
    $\omega \nvDash P$ or $\omega \vDash Q$.
  \item $\omega \vDash P \leftrightarrow Q$ iff 
    ($\omega \vDash P$ and $\omega \vDash Q$) or 
    ($\omega \nvDash P$ and $\omega \nvDash Q$).
  \item $\omega \vDash \forall x P$ iff 
    $\omega^d_x \vDash P$ for all $d \in \mathds{R}$. 
  \item $\omega \vDash \exists x P$ iff 
    $\omega^d_x \vDash P$ for some $d \in \mathds{R}$.
\end{itemize}
\vspace{-.5cm}
\textbf{State Modification:}
\begin{align*}
  \omega^d_x(y) = 
  \begin{cases} 
    d \text{, if } y \text{ is the modified variable } x \\
    \omega(y) \text{, if } y \text{ is another variable}
  \end{cases}
\end{align*}

If $\omega \vDash P$, then we say that $P$ is true at $\omega$ 
or that $\omega$ is a model of $P$. 
Otherwise, i.e., if $\omega \nvDash P$, we say that $P$ is false at $\omega$. 
A formula $P$ is \ctextbf{valid}, written $\vDash P$, 
iff $\omega \vDash P$ for all states $\omega$. 
Formula $P$ is called \ctextbf{satisfiable} 
iff there is a state $\omega$ such that $\omega \vDash P$. 
Formula $P$ is \ctextbf{unsatisfiable} 
iff there is no such $\omega$. 
The set of states in which formula $P$ is true is written 
$\llbracket P \rrbracket = \{ \omega : \omega \vDash P \}$ .

\textbf{Semantics of Continuous Program:}
The state $\nu$ is reachable from initial state $\omega$ 
by the continuous program $x_1' = e_1, \dots, x_n' = e_n \& Q$
iff there is a \ctextbf{solution} $\varphi$ of some duration $r \geq 0$
along the program from state $\omega$ to state $\nu$, i.e., 
a function $\varphi:[0, r] \rightarrow \mathscr{S}$ such that:
\vspace{-.5cm}
\begin{itemize}
  \item initial and final states match: $\varphi(0) = \omega, \varphi(r) = \nu$;
  \item $\varphi$ respects the differential equations: For each variable $x_i$, 
    the value $\varphi(\zeta)\llbracket x_i \rrbracket$ of $x_i$ at 
    state $\varphi(\zeta)$ is continuous in $\zeta$ on $[0, r]$ and,
    if $r > 0$, has a time-derivative of value 
    $\varphi(\zeta)\llbracket e_i \rrbracket$ at each time $\zeta \in [0, r]$, 
    i.e.,
    \begin{align*}
      \frac{d\varphi(t)(x_i)}{dt}(\zeta) = \varphi(\zeta)\llbracket e_i \rrbracket
    \end{align*}
  \item the value of other variables $y \notin \{x_1, \dots, x_n\}$ remains 
    constant throughout the continuous evolution, that is 
    $\varphi(\zeta)\llbracket y \rrbracket = \omega\llbracket y \rrbracket$ 
    for all times $\zeta \in [0, r]$;
  \item and $\varphi$ respects the evolution domain at all times: 
    $\varphi(\zeta) \in \llbracket Q \rrbracket$ for each $\zeta \in [0, r]$.
\end{itemize}

\crule
\subsection{Choice and Control}

\subsubsection{Hybrid Programs}

\clanewdef{Hybrid Program (HP)}{ 
  HPs are defined by the following grammar 
  ($\alpha, \beta$ are HPs, $x$ is a variable, $e$ is a term possibly 
  containing $x$, and $Q$ is a formula of first-order logic of 
  real arithmetic):
  \begin{align*}
    \alpha, \beta ::= x := e | ?Q | x' = f(x) \& Q | \alpha \cup \beta 
      | \alpha ; \beta | \alpha^*
  \end{align*}
  \vspace{-1cm}
  \begin{itemize}
    \item \textbf{Sequential composition: $\alpha ; \beta$}
    \item \textbf{Test: $?Q$}
    \item \textbf{Nondeterministic choice: $\alpha \cup \beta$}
    \item \textbf{Nondeterministic repeat: $\alpha^*$}
  \end{itemize}
}

The meaning of an HP $\alpha$ is given by a \ctextbf{reachability relation} 
(not a function) $\llbracket \alpha \rrbracket \subseteq \mathscr{S} 
\times \mathscr{S}$ on states.
So $(\omega, \nu) \in \llbracket \alpha \rrbracket$ means that final state $\nu$
is reachable from initial state $\omega$ by running HP $\alpha$.

\clanewdef{Transition semantics of HPs}{
  Each HP $\alpha$ is interpreted semantically as a binary reachability 
  relation $\llbracket \alpha \rrbracket \subseteq \mathscr{S} \times \mathscr{S}$
  over states, defined inductively by:
  \vspace{-.5cm}
  \begin{enumerate}
    \item $\llbracket x := e \rrbracket = \{ (\omega, \nu): \nu = \omega 
      \text{ except that } \nu\llbracket x \rrbracket = 
      \omega\llbracket e \rrbracket \}$.
    \item $\llbracket ? Q \rrbracket = \{ (\omega, \omega): 
      \omega \in \llbracket Q \rrbracket\}$.
    \item $\llbracket x' = f(x) \& Q \rrbracket = 
      \{ (\omega, \nu): \varphi(0) = \omega \text{ except at } x' 
      \text{ and } \varphi(r) = v$ for a solution 
      $\varphi:[0, r] \rightarrow \mathscr{S}$ 
      of any duration $r$ satisfying $\varphi \vDash x' = f(x) \wedge Q \}$.
    \item $\llbracket \alpha \cup \beta \rrbracket = 
      \llbracket \alpha \rrbracket \cup \llbracket \beta \rrbracket$.
    \item $\llbracket \alpha;\beta \rrbracket = 
      \llbracket \alpha \rrbracket \circ \llbracket \beta \rrbracket = 
      \{(\omega, \nu): (\omega, \mu) \in \llbracket \alpha \rrbracket, 
      (\mu, \nu) \in \llbracket \beta \rrbracket\}$
    \item $\llbracket \alpha^{*} \rrbracket = \llbracket \alpha \rrbracket^{*} = 
      \bigcup_{n\in\mathds{N}}\llbracket \alpha^{n}\rrbracket$ with 
      $\alpha^{n+1} \equiv (\alpha^{n};\alpha)$ and
      $\alpha^{0} \equiv ?true$.
  \end{enumerate}
}

\clanewdef{Transition semantics of ODEs}{
  $\llbracket x' = f(x) \& Q \rrbracket = \{(\omega, \nu): \varphi(0) = \omega 
  \text{ except at } x' \text{ and } \varphi(r) = \nu \text{ for a solution }
  \varphi:[0, r] \rightarrow \mathscr{S} \text{ of any duration } r 
  \text{ satisfying } \varphi \vDash x' = f(x) \wedge Q\}$ where
  $\varphi \vDash x' = f(x) \wedge Q$, iff for all times 
  $0 \leq z \leq r: \varphi(z) \in \llbracket x' = f(x) \wedge Q \rrbracket$
  with $\varphi(z)(x') \stackrel{\text{def}}{=} \frac{d\varphi(t)(x)}{dt}(z)$ 
  and $\varphi(z) = \varphi(0) \text{ except at } x, x'$.
}

\subsubsection{Hybrid Program Design}

\textbf{Principle 1}: Verification models benefit from nondeterminism,
because if highly nondeterministic models have no unsafe behaviors then all 
more specific refinements have no unsafe behaviors either.

\crule